\section{Conclusions}

The rSLDS model, in all the cases that we examined, was able to capture the dynamics of complex canonical dynamical systems, at least under certain conditions. When the dynamics of the underlying system feature a stable fixed point or an unstable orbit, the model is only able to reproduce the dynamics if the noise on the observations is sufficiently small. However, if the system exhibits a stable orbit or limit cycle, the rSLDS model is particularly effective at reproducing the dynamics of the underlying system.

Another potential downside of this model is that it is only able to reproduce the dynamics associated with a single trajectory, at least with the current fitting methods. Since the model depends on time series data, we cannot append multiple trajectories in the same system to fit the model for broader initial conditions. This makes the method limited in learning the global dynamics and showing where the boundaries of stability are especially the position of unstable fixed points and periodic orbits.

\begin{comment}
The method can be very successfully applied to systems with a periodic orbit or strange attractor and the goal is to find the dynamics of this orbit. We can remark that this method is very limited in its ability to give an estimation on the linear estimation of the global dynamics. Unlike the SINDY algorithm \cite{} it is currently not possible to learn from data from multiple trajectories. This makes the method limited in learning the global dynamics and showing where the boundaries of stability are especially the position of unstable fixed points and periodic orbits. This makes the method only applicable for initial conditions that lead to periodic or bounded trajectories. The extension of implementing the ability to handle multiple trajectories will enhance the range of problems this method can be applied to.
\end{comment}
