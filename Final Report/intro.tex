%%%%%%%%%%%%%%%%%%%%%%%%%%%%%%%%%%%%%%%%%%%%%%%%%%%%%
% When you are done editing, comment out everything but the text and input into the main document.
%%%%%%%%%%%%%%%%%%%%%%%%%%%%%%%%%%%%%%%%%%%%%%%%%%%%%

 \begin{comment}
\documentclass{amsart}

\usepackage{packages} % Look here for packages before you add more, place all figures in Figures folder
\usepackage{macros} % Check here for macros before you add more
\newcommand{\prob}{\mathbb{P}}
\newcommand{\iid}{\overset{\mathrm{iid}}{\sim}}

\usepackage{setspace}
	\doublespacing

\begin{document}

 \end{comment}
\section{Introduction}

Consider a basketball player coming the court. They arrive at the three point line and have a decision to make: should they cut to the basket or run to the left corner. Depending on the positions of other players, the state in the game, or whether they are playing at home or away, the player might make a different decision. Scenarios like these produce data generated by complex, nonlinear dynamical systems. Instead of trying to produce the exact equations that lead to dynamics like those described above, we can instead understand these systems by decomposing them into multiple, simpler dynamical systems.

\subsection{Switching Linear Dynamical Systems}

Switching linear dynamical systems(SLDSs) are used to split up complex, nonlinear dynamical system data into a set of linear models. By fitting the data to an SLDS, a nonlinear representation is learned. Given sufficient data and an adequate number of linear models, it can learn the global dynamics of the system.

An SLDS has the following components for each time step $t = 1,2, \dots, T$: a discrete latent state $z_t$, a continuous latent state $x_t$, and an observation $y_t$. The underlying state $z_t$ is in the set $\brac{1,2,\dots,K}$ and is driven by a Markov process:
\[\prob(z_{t+1} = j \mid z_t = i) = \pi_{ij}, \]
where $\brak{\pi_{ij}}_{K\times K}$ is the transition matrix of the process. The underlying continuous state $x_t$ is in $\R^M$ and is driven by conditionally linear dynamics, dependent on the discrete latent state $z_{t+1}$:
\[x_{t+1} = A_{z_{t+1}} x_t + b_{z_{t+1}} + v_t, \]
where $A_k\in\R^{M\times M}$, $b_k\in \R^M$, and $v_t \iid \mathcal{N}(0,Q_{z_{t+1}})$, where $Q_k\in\R^{M\times M}$ is the covariance matrix of the normal distribution. Finally, the observation $y_t$ is generated from the corresponding discrete continuous and discrete latent states:
\[y_t = C_{z_{t}} x_t + d_{z_{t}} + w_t,\]
where $C_k\in\R^{N\times M}$, $d_k\in \R^N$, and $w_t \iid \mathcal{N}(0,S_{z_{t}})$, where $S_k\in\R^{N\times N}$ is the covariance matrix of the normal distribution.

\subsection{Fitting SLDSs}

\textcolor{red}{Talk about stick-breaking ... and Polya-gamma ...}

\subsection{Recurrent Switching Linear Dynamical Systems (rSLDSs)}

If a switch in the discrete state of a system should occur when the trajectory enters (or leaves) a particular region, the model will be unable to capture this behavior since the discrete state is a function only of the previous discrete state. To rectify this, we instead consider a recurrent switching linear dynamical system (rSLDS). 

\subsection{Conclusions from the Paper}



\textcolor{red}{Discuss the mathematical or scientific question that is addressed in the paper. Cite at least two other articles}
\begin{comment} 
\end{document}
 \end{comment}