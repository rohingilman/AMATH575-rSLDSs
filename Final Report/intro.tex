%%%%%%%%%%%%%%%%%%%%%%%%%%%%%%%%%%%%%%%%%%%%%%%%%%%%%%%
% Make sure to add any new macros to the final report 
% to ensure that they do not conflict with existsing
% ones.
%%%%%%%%%%%%%%%%%%%%%%%%%%%%%%%%%%%%%%%%%%%%%%%%%%%%%%%

\section{Introduction}

Consider a basketball player coming the court. They arrive at the three point line and have a decision to make: should they cut to the basket or run to the left corner. Depending on the positions of other players, the state in the game, or whether they are playing at home or away, the player might make a different decision. Scenarios like these produce data generated by complex, nonlinear dynamical systems. Instead of trying to produce the exact equations that lead to dynamics like those described above, we can instead understand these systems by decomposing them into multiple, simpler dynamical systems.

\subsection{Switching Linear Dynamical Systems}

Switching linear dynamical systems(SLDSs) are used to split up complex, nonlinear dynamical system data into a set of linear models. By fitting the data to an SLDS, a nonlinear representation is learned. Given sufficient data and an adequate number of linear models, it can learn the global dynamics of the system.

An SLDS has the following components for each time step $t = 1,2, \dots, T$: a discrete latent state $z_t$, a continuous latent state $x_t$, and an observation $y_t$. The underlying state $z_t$ is in the set $\brac{1,2,\dots,K}$ and is driven by a Markov process:
\[\prob(z_{t+1} = j \mid z_t = i) = \pi_{ij}, \]
where $\brak{\pi_{ij}}_{K\times K}$ is the transition matrix of the process. The underlying continuous state $x_t$ is in $\R^M$ and is driven by conditionally linear dynamics, dependent on the discrete latent state $z_{t+1}$:
\[x_{t+1} = A_{z_{t+1}} x_t + b_{z_{t+1}} + v_t, \]
where $A_k\in\R^{M\times M}$, $b_k\in \R^M$, and $v_t \iid \mathcal{N}(0,Q_{z_{t+1}})$, where $Q_k\in\R^{M\times M}$ is the covariance matrix of the normal distribution. Finally, the observation $y_t$ is generated from the corresponding discrete continuous and discrete latent states:
\[y_t = C_{z_{t}} x_t + d_{z_{t}} + w_t,\]
where $C_k\in\R^{N\times M}$, $d_k\in \R^N$, and $w_t \iid \mathcal{N}(0,S_{z_{t}})$, where $S_k\in\R^{N\times N}$ is the covariance matrix of the normal distribution.

The library of the linear systems involved in the SLDS is denoted by
\[\theta = \brac{\pi_k, A_k, b_k, Q_k, C_k, d_k, S_k \mid k = 1, \dots, K}.\]

\subsection{Fitting SLDSs}

\textcolor{red}{Talk about stick-breaking ... and Polya-gamma ...}

\subsection{Recurrent Switching Linear Dynamical Systems (rSLDSs)}

If a switch in the discrete state of a system should occur when the trajectory enters (or leaves) a particular region, the model will be unable to capture this behavior since the discrete state is a function only of the previous discrete state. To rectify this, we instead consider a recurrent switching linear dynamical system (rSLDS). 

The key difference between an SLDS and an rSLDS is in the update of the  discrete latent state, $z_t$. Instead of depending only on the previous discrete state, the update also depends on the continuous latent variable $x_t$. We can see this difference in dependencies Figure \ref{rSLDS}.

\begin{figure}[h!]
	\centering
	\includegraphics[width=0.5\textwidth]{rSLDS_diagram.png}
	\caption{The black arrows represent the dependencies of a tradidional SLDS, and the red arrows represent the added dependencies of an rSLDS.}
	\label{rSLDS}
\end{figure}

\subsection{Fitting rSLDSs} \textcolor{red}{Differences in fitting SLDS vs rSLDS}

\subsection{Conclusions from the Paper}

In the paper (\textcolor{red}{Cite Linderman Here}), the authors use three examples to demonstrate the capabilities (and limitations) of SLDSs and rSLDSs. These examples are data produced by an actual rSLDS (Synthetic NASCAR), data simulated from a well-studied classical dynamical system (Lorenz Attractor), and real data (Basketball Player Trajectories).

The models performed reasonably well on the synthetic data. Figure \ref{trueNascar} shows the true latent dynamics of the synthetic NASCAR simulation.
\begin{figure}[h!]
	\centering
	\includegraphics[width=0.35\textwidth]{paper_fig1.png}
	\caption{True dynamics of synthetic NASCAR}
	\label{trueNascar}
\end{figure}

Figure \ref{Nascargen} shows the states generated by both the SLDS and rSLDS fitting of the data. We can see that the rSLDS far outperformed the SLDS. This makes sense because the data was actually generated from an rSLDS, which the SLDS model will not be able to capture. Although this is just simulated data, this proof of concept demonstrates that if the data does indeed come from an SLDS or rSLDS, the approach from the paper will generate a reasonable interpretation of that data.

\begin{figure}[h!]
	\centering
	\begin{subfigure}[b]{0.35\textwidth}
		\includegraphics[width=\textwidth]{paper_fig3.png}
		\caption{SLDS}
	\end{subfigure}
	\begin{subfigure}[b]{0.35\textwidth}
		\includegraphics[width=\textwidth]{paper_fig4.png}
		\caption{rSLDS}
	\end{subfigure}
	\caption{Generated states of fit data}
	\label{Nascargen}
\end{figure}

\textcolor{red}{Lorenz data}

\textcolor{red}{Basketball data}

\subsection{New Methods}